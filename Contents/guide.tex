\scseparatedfragment[\scnidtf{Руководство разработчика Стандарта OSTIS}]{Краткое руководство разработчика Стандарта OSTIS}

\begin{SCn}

\scnsectionheader{\currentname}
\begin{scnitemize}
	\item \textbf{Что нужно знать для старта}
	\begin{scnitemizeii}
		\item Изучить Youtube-канал с вводными видео о \textit{Технологии OSTIS}. Для начала рекомендуются к просмотру \textit{Введение в Технологию OSTIS} и \textit{Введение в SC-код};
		\item Изучить опубликованную версию \textit{Стандарта OSTIS-2021}. Для начала работы целесообразно ознакомиться с разделами:
		\begin{scnitemizeiii}
			\item Принципы построения и структура \textit{Стандарта OSTIS};
			\item Описание внутреннего языка \textit{ostis-систем};
			\item Предметная область и онтология внешних идентификаторов знаков, входящих в информационные конструкции внутреннего языка \textit{ostis-систем};
			\item Описание языка графического представления информационных конструкций в \textit{ostis-системах};
			\item Описание языка линейного представления информационных конструкций в \textit{ostis-системах};
			\item Описание языка структурированного представления информационных конструкций в \textit{ostis-системах};
			\item Описание общих принципов оформления внутреннего и внешнего представления информационных конструкций в \textit{ostis-системах};
			\item Предметная область и онтология смыслового представления информации.	
		\end{scnitemizeiii}
		Для навигации используйте оглавление изданной версии \textit{Стандарта OSTIS-2021}.
		\item Целесообразно внимательно ознакомиться с актуальной версией \textit{Оглавления Стандарта OSTIS}, которая значительно больше, чем оглавление опубликованной версии \textit{Стандарта OSTIS-2021}. Наиболее актуальная и полная версия \textit{Оглавления} формируется при компиляции текста \textit{Стандарта из однодников} (см. \textit{Основные принципы работы с исходниками Стандарта OSTIS посредством Github}). В актуальную версию включены как разделы, уже имеющие некоторое наполнение, так и разделы, которые планируются к разработке в будущем (планы развития стандарта). Таким образом, любой желающий может выбрать наиболее интересное для себя направление для подключения к разработке \textit{Стандарта}.;
		\item Целесообразно внимательно ознакомиться с перечнем ключевых знаков, приведенный в \textit{Титульной спецификации Стандарта}, и синонимов к ним, что позволит лучше понять терминологию, используемую в рамках стандарта, а следовательно и сам стандарт.
	\end{scnitemizeii}
	\item \textbf{Основные принципы работы с исходными текстами Стандарта OSTIS посредством Github}\\
	\begin{scnitemizeii}
		\item \textbf{Начало работы с Git. Клонирование репозитория}
		\begin{scnitemizeii}
			\item Устанавливаем GitHub Desktop, SmarGit или любой другой графический клиент для Git. Можно работать и через терминал. Дальнейшие шаги показаны на примере GitHub Desktop;
			\item После установки в появившемся окне нажимаем “Sign in to GitHub.com”;
			\item В открывшемся окне браузера вводим в форму свои данные, как при регистрации, и нажимаем “Sign in”. Или же создаем новый аккаунт;
			$\newline$
			\scnfilescg{Contents/guide_image/image14}
			\item Если браузер запросит, то подтвердить, что нужно “Открыть приложение GitHub Desktop”;
			$\newline$
			\scnfilescg{Contents/guide_image/image15}
			$\newline$
			\scnfilescg{Contents/guide_image/image16}
			\item Далее регистрационные данные перенесутся в форму конфигурации (настроек) Git - нажимаем “Finish”;
			$\newline$
			\scnfilescg{Contents/guide_image/image17}
			\item Далее видим начальное окно GitHub Desktop;
			$\newline$
			\scnfilescg{Contents/guide_image/image18}
			\textbf{“Create a tutorial repository…“} - создать обучающий репозиторий.\\
			
			\textbf{“Clone repository from the Internet…“} - клонировать (скопировать/скачать) репозиторий из GitHub к себе на компьютер.\\
			
			\textbf{“Create a New Repository on your hard drive…“} - создать новый репозиторий на вашем жестком диске (на вашем компьютере) и добавить систему Git в проект.\\
			
			\textbf{“Add an Existing Repository from your hard drive…“} - добавить на GitHub репозиторий, который уже есть на вашем компьютере и использует Git.\\
			
			Справа будут отображаться ваши репозитории, которые уже загружены на GitHub, но если только что зарегистрировались, то список будет пуст.
			
			
			\item В меню выбираем File > Clone Repository;
			$\newline$
			\scnfilescg{Contents/guide_image/image19}
			\item Выбираем URL и вставляем адрес проекта по разработке Стандарта Технологии OSTIS:  \uline{https://github.com/ostis-ai/ostis-standard}. Указываем папку назначения для проекта;
			$\newline$
			\scnfilescg{Contents/guide_image/image20}
			\item Репозиторий успешно склонирован, теперь у вас есть локальная копия. 
		\end{scnitemizeii}
		\item \textbf{Работа с TeX, компиляция проекта в PDF}
		\begin{scnitemizeii}
			\item Непосредственно в рамках репозитория есть файл readme.md, в котором описана инструкция для установки дистрибутива Tex и компиляции проекта для ОС на базе Linux. Для ОС Windows установка осуществляется аналогично, далее процесс установки и настройки будет показан на примере дистрибутива MiKTex и среды TeXstudio;		
			\item Для работы с TeX необходимо скачать и установить дистрибутив TeX, например, MiKTeX или Texlive. Инструкция по установке MiKTeX здесь. Для работы под ОС на базе Linux рекомендуется использовать дистрибутив Texlive, при этом обязательно установить полную версию дистрибутива со всеми пакетами (texlive-full);
			\item Далее устанавливаем любой удобный редактор LaTeX, например TeXstudio. Руководство пользователя TeXstudio находится на официальном сайте;
			\item После установки MiKTex и Texstudio, запускаем Texstudio;
			\item Файл -> Открыть (Ctrl + O). Находим папку с проектом, указанную на 8 шаге. Выбираем book.tex;
			$\newline$
			\scnfilescg{Contents/guide_image/image21}
			\item Для компиляции и одновременного просмотра проекта нажимаем F5 (двойная зеленая стрелочка в меню) или же F6 для компиляции (одинарная зеленая стрелка).
		\end{scnitemizeii}
		\uline{Примечания:}\\
		\begin{scnitemizeii}
			\item Для компиляции полной pdf-версии стандарта с полным оглавлением и вертикальными фоновыми линиями необходимо произвести полную компиляцию 2-3 раза;
			\item В процессе компиляции стандарта могут в большом количестве возникать  ошибки (например, “no line here to end”), связанные с недоработками в реализации текущей версии команд LaTex, используемых при разработке стандарта. Эти ошибки не влияют на результат компиляции. Если PDF не отображается автоматически, его можно увидеть, выполнив команду View (F7);
			\item Для корректной компиляции библиографических источников иногда требуется при первой сборке выполнить команду latexmk. Ниже показано расположение команды в меню в среде TexStudio.
			$\newline$
			\scnfilescg{Contents/guide_image/image22}
		\end{scnitemizeii}
		\item \textbf{Внесение локальных изменений в исходный текст Стандарта OSTIS}
		Все изменения обязательно делаются в отдельной ветке репозитория.
		После каждого логически законченного изменения делается коммит. В сообщении коммита предпочтительно на английском языке пишется какие разделы стандарта были изменены и в чем заключается суть изменений.
		\item \textbf{Внесение изменений в основной репозиторий Стандарта OSTIS}
		Поскольку репозиторий \textit{Стандарта OSTIS} является открытым, и принять участие в работе над \textit{Стандартом} потенциально может любой желающий, то работа с репозиторием осуществляется через механизм fork и pull-request. Почитать об этом подробнее можно здесь.\\
		
		Каждый pull-request должен пройти рецензирование как минимум:\\
		\begin{scnitemizeii}
			\item хотя бы одним редактором того раздела, в который вносятся изменения;
			\item хотя бы одним членом \textit{Редколлегии Стандарта OSTIS}.
		\end{scnitemizeii}
		Рецензентов автор pull-request-a может назначить самостоятельно, чтобы ускорить процесс рецензирования и исключить необходимость всем участникам процесса регулярно просматривать все приходящие pull-request-ы.   \\
		
		В процессе работы и обсуждения pull-request-а  допускается наличие временных коммитов с временными условными сообщениями. После принятия pull-request-а история коммитов приводится в порядок при помощи force push и осуществляется слияние (merge). Слияние выполняет член \textit{Редколлегии Стандарта OSTIS} после проверки истории коммитов.\\
		
		Для опытных авторов, внесших значительный вклад в развитие \textit{Стандарта} может рассматриваться вопрос о добавлении в основной репозиторий в качестве коллаборатора. В этом случае работа ведется в отдельной ветке и pull-request делается из этой ветки в основную ветку (master).\\
	\end{scnitemizeii}
	\item \textbf{Основные принципы работы с исходными текстами Стандарта OSTIS посредством Github}
	\begin{scnitemizeii}
		\item \textbf{Основные принципы работы с командами scn-tex}
		\begin{scnitemizeiii}
			\item Для того, чтобы иметь возможность, с одной стороны, формировать читабельный текст \textit{Стандарта OSTIS}, пригодный для его публикации в виде книги или включения его фрагментов в другие печатные издания, а с другой стороны, иметь формальный исходный текст \textit{Стандарта OSTIS}, который может быть автоматически протранслирован в базу знаний любой \textit{ostis-системы}, был выбран вариант разработки исходных текстов \textit{Стандарта} с использованием набора команд, разработанных на основе языка LaTex;
			\item Предлагаемый набор команд условно называется \textit{scn-tex}, поскольку в его основу положена идея того, чтобы разработчик писал исходный текст максимально приближенно к тому, как он будет видеть результат компиляции этого исходного текста (в SCn-коде) и при этом максимально был избавлен от необходимости учитывать особенности языка LaTex в работе. Таким образом, весь исходный текст стандарта формируется исключительно с использованием набора команд \textit{scn-tex}, запрещается использовать любые другие команды для форматирования текста, изменения шрифта, вставки внешних файлов и т.д. В рамках естественно-языковых фрагментов, входящих в состав стандарта, допускается использование команд LaTex для вставки специальных символов и математических формул. Для добавления файлов изображений в текст стандарта используются только команды \textit{scn-tex}. Также только команды \textit{scn-tex} используются для формирования нумерованных и маркированных списков, добавления закрывающих и открывающих скобок различного вида (кроме круглых);
			\item Для выделения курсивом \textit{идентификаторов} в рамках естественно-языковых фрагментов, входящих в состав стандарта, используется только команда $\backslash$textit\{\}. Для выделения полужирным курсивом используется комбинация команд $\backslash$textbf\{$\backslash$textit\{\}\} (именно в таком порядке);
			\item Каждая команда из набора \textit{scn-tex} начинается с префикса $\backslash$scn, после которого идет имя команды, примерно описывающее то, как связан текущий отображаемый фрагмент текста с описываемой сущностью.\\
			\uline{Например:}\\
			$\backslash$scnrelfrom - дуга ориентированного отношения, которая выходит из описываемой сущности в другую сущность;\\
			$\backslash$scnrelto - дуга ориентированного отношения, которая входит в описываемую сущность другой;\\
			$\backslash$scnnote - естественно-языковое примечание к описываемой сущности.\\
			Полный перечень команд можно увидеть в файле scn.tex, а примеры использования команд каждого типа - в исходных текстах стандарта.
			\item Для формирования отступов в SCn-тексте используется команда $\backslash$scnaddlevel\{x\}, где х - число уровней, на которое необходимо сместиться. Число х должно быть целым, но не обязательно положительным, как правило после смещения на определенное число уровней вправо следует смещение на то же число уровней влево. Для сброса уровня до нуля (левый край страницы) можно использовать команду $\backslash$scnresetlevel.\\
			\uline{Пример:}\\
			Исходный код:
			$\newline$
			\scnfilescg{Contents/guide_image/image1}
			Результат компиляции:\\
			$\newline$
			\scnfilescg{Contents/guide_image/image2}
			\end{scnitemizeiii}
			\item \textbf{Добавление разделов (отдельный файл и label)}\\
			Исходный текст каждого раздела \textit{Стандарта} настоятельно рекомендуется записывать в отдельном tex-файле для последующего удобства коллективной работы. Если раздел имеет большой размер и/или делится на фрагменты, то целесообразно разделить исходный текст раздела на несколько tex-файлов.
			Структура директорий с tex-файлами обычно повторяет иерархию разделов в Оглавлении \textit{Стандарта OSTIS}.\\
			Добавление нового раздела осуществляется в 3 шага:
			\begin{scnitemizeiii}
				\item Добавление имени раздела и его спецификации в Оглавление Стандарта OSTIS. В зависимости от уровня раздела в оглавлении для этого используются команды $\backslash$scchapter, $\backslash$scsection, $\backslash$scsubsection, $\backslash$scsubsubsection. $\backslash$scparagraph, $\backslash$scsubparagraph.\\
				Если есть необходимость специфицировать раздел прямо в оглавлении, то соответствующие команды размещаются в квадратных скобках сразу после команды scsection и других аналогичных (см. пример ниже) перед фигурными скобками, в которых записывается имя раздела;
				\item Добавление метки (label), соответствующей данному разделу. Метка нужна для того, чтобы иметь возможность обратиться к имени раздела в других местах текста не копируя его явно. Это удобно в случае переименования раздела.\\
				Имя метки рекомендуется формировать на основе англоязычных идентификаторов ключевых знаков данного раздела, например:\\  \textit{Предметная область и онтология базовых интерпретаторов логико-семантических моделей ostis-систем} -> sd\_interpreters (SD - subject domain). Как правило, имя метки совпадает с именем tex-файла, содержащего исходный текст раздела.\\
				Для вставки имени раздела в именительном падеже в какое-либо место текста необходимо использовать только команду $\backslash$nameref\{<имя метки>\}, запрещается набирать имя раздела вручную обычным текстом. Исключение составляют ситуации, когда имя раздела необходимо использовать в другом падеже или необходимо использовать только часть имени. Использование меток позволяет в случае переименования раздела автоматически изменять имя раздела во все местах текста, где осуществляется отсылка к этому имени;
				\item Подключение tex-файла с исходным текстом раздела при помощи команды include в рамках основного файла проекта (book.tex) и input в других местах.
			\end{scnitemizeiii}
			\uline{Пример:}\\
			$\newline$
			\scnfilescg{Contents/guide_image/image3}
			\item \textbf{Временное удаление раздела}\\
			Во время локальной работы над конкретным разделом нецелесообразно компилировать весь текст стандарта, поскольку полная компиляция занимает достаточно много времени. Для того чтобы ускорить компиляцию проекта можно временно исключить те разделы или группы разделов, которые в данный момент не важны. Проще всего это сделать закомментировав строку, в которой подключается текст соответствующего раздела или группы разделов: \\
			$\newline$
			\scnfilescg{Contents/guide_image/image4}
			или\\
			$\newline$
			\scnfilescg{Contents/guide_image/image5}
			\item \textbf{Добавление библиографии}\\
			\begin{scnitemizeiii}
			\item \textbf{Суть задачи}\\
			Одним из вариантов полезной и не очень объемной задачи в рамках проекта по развитию \textit{Стандарта OSTIS} является дополнение библиографии \textit{Стандарта}. Задачи такого плана хорошо подойдут для тех, кто хотел бы попробовать свои силы в работе над Стандартом и помогут адаптироваться к выполнению более сложных задач.\\
			Библиография в данном случае трактуется в самом широком смысле и включает не только перечень собственно библиографических источников, но и их достаточно подробную спецификацию, перечень проектов и систем, изучение которых представляет интерес в рамках \textit{Технологии OSTIS}, перечень конкретных персон, работавших  и работающих в областях, смежных с \textit{Технологией OSTIS} и т.д.\\
			Конкретными задачи по расширению библиографии Стандарта OSTIS являются:
			\begin{scnitemizeii}
				\item Привязка глав (разделов, параграфов) книг или конкретных статей по частям или целиком к соответствующим понятиям или конкретным сущностям из стандарта;\\
				\uline{Пример:}\\
				$\newline$
				\scnfilescg{Contents/guide_image/image6}
				$\newline$
				\scnfilescg{Contents/guide_image/image7}
				\item Выделение конкретных цитат или перефразированных фрагментов каких-либо текстов и их привязка к соответствующим понятиям стандарта;\\
				\uline{Пример:}\\
				$\newline$
				\scnfilescg{Contents/guide_image/image8}
				$\newline$
				\scnfilescg{Contents/guide_image/image9}
				\item Дополнение списка синонимичных идентификаторов, пояснений и примечаний для понятий стандарта с обязательным указанием того источника, откуда взят соответствующий идентификатор или фрагмент текста. В данном случае важно учитывать, чтобы описываемое понятие одинаково трактовалось как в рамках стандарта, так и в рамках того источника, откуда берется информация и не возникало противоречий в рамках стандарта. В стандарте могут быть описаны альтернативные точки зрения или противоречащие друг другу взгляды на одну и ту же проблему, но это должно быть явно помечено как противоречие;
				\item Описание сравнительного анализа различных сущностей, с указанием их сходств и отличий. Это актуально как для понятий, непосредственно описываемых в рамках каких-либо разделов стандарта, так и для сравнения \textit{Технологии OSTIS} с другими близкими технологиями, в частности, сравнения языков, разрабатываемых в рамках \textit{Технологии OSTIS}, с другими аналогичными языками, сравнения понятий, исследуемых в рамках \textit{Технологии OSTIS}, с близкими понятиями из других технологий и т.д.
			\end{scnitemizeii}
			\item \textbf{Как добавить свой библиографический источник}\\
			Для описания библиографических источников используется средство BibTex.\\ 
			Для добавления нового библиографического источника необходимо выполнить следующие шаги:
			\begin{scnitemizeii}
				\item Убедиться, что нужный источник еще не присутствует в файле biblio.bib, который находится в репозитории с исходными текстами \textit{Стандарта OSTIS}. В настоящее время все библиографические источники изначально описываются в этом файле;
				\item Добавить в файл biblio.bib описание библиографического источника в соответствии с форматом описания BibTex. Для помощи в оформлении можно использовать различные бесплатные средства, например, сервис doi2bib позволяет сгенерировать bib-описание на основе идентификатора DOI, кроме того, многие онлайн-библиотеки позволяют выгрузить описание нужного источника в формат BibTex;
				\item Каждому источнику в соответствии с форматом BibTex присваивается некоторое условное имя (цитатный ключ или просто ключ), по которому затем можно процитировать соответствующий источник. В рамках \textit{Стандарта OSTIS} рекомендуется цитатные ключи источников в формате BibTex формировать путем транслитерации в латинский алфавит фамилии первого автора и добавления года издания источника, например:\\
				Trudeau1993\\
				Golenkov2011\\
				Если при этом возникает неоднозначность, связанная с тем, что существует несколько работ того же автора в один год, то в конце ключа рекомендуется добавлять строчные латинские буквы a, b, c и так далее, например:\\
				
				Gribova2015a\\
				Gribova2015b\\
				
				При формировании ключа для электронного источника или коллективной публикации, где невозможно выделить ключевого автора, рекомендуется формировать ключ из 1-2 английских слов или аббревиатур, позволяющих однозначно идентифицировать соответствующий источник. При использовании нескольких слов их можно соединять знаком нижнее подчеркивание, пробелы в ключах запрещены. При необходимости в конце ключа можно добавлять год издания. Например:\\
				
				IMS (библиографическая ссылка на сайт Метасистемы IMS.ostis)
				CYPHER (библиографическая ссылка на сайт с описанием языка Cypher)
				AIDictionary1992 (библиографическая ссылка на Словарь по исусственному интеллекту 1992 года издания);
				\item Для добавленного источника необходимо описать его идентификатор, который далее будет использоваться в рамках текста Стандарта. Это делается при помощи BibTex поля shorthand, например (см. Правила идентификации библиографических источников):\\
				
				shorthand = \{Trudeau.R.J.IntroGT-1993кн\}\\
				shorthand = \{Duchi.J..AdaptiveSubgradMethods-2011ст\}\\
				shorthand = \{Грибова.В.В..БазоваяТРИСОП-2015ст\}
				
				\item Далее этот идентификатор может использоваться как в формальном тексте, также как и идентификатор любой другой сущности, так и в рамках естественно-языкового текста. Для автоматической вставки идентификатора библиографического источника в формальный либо естественно-языковой текст используется команда $\backslash$scncite\{<цитатный ключ>\}, например:\\
				\uline{Исходный код:}\\
				$\newline$
				\scnfilescg{Contents/guide_image/image10}
				\uline{Результат компиляции:}\\
				$\newline$
				\scnfilescg{Contents/guide_image/image11}
				\item Для каждого источника крайне желательно добавить его краткую аннотацию. Это делается при помощи BibTex поля annotation, например:\\
				
				annotation = \{В этой книге представлены исследования по внедрению концептуальных основ, стратегий, методов, методологий, информационных платформ и моделей для разработки современных систем, основанных на знаниях\} \\
				В рамках аннотации допускается использование средств форматирования естественно-языковых текстов, принятых в рамках Стандарта OSTIS, например, выделение курсивом или полужирным курсивом.
				Для вставки аннотации в формальный scn-текст используется команда $\backslash$scnciteannotation\{<цитатный ключ>\}, например:\\
				\uline{Исходный код:}\\
				$\newline$
				\scnfilescg{Contents/guide_image/image12}
				\uline{Результат компиляции:}\\
				$\newline$
				\scnfilescg{Contents/guide_image/image13}
			\end{scnitemizeii}
			\item \textbf{Правила идентификации библиографических источников}
			Идентификаторы статей, книг и других печатных работа строятся следующим образом:\\
			\begin{scnitemizeii}
				\item Пишется фамилия первого автора на том языке, на котором опубликована данная работа. Затем через точку ставятся инициал(-ы) первого автора;
				\item Если работа опубликована с участием только одного автора, то ставится одна точка, если нескольких - две точки;
				\item Пишется первое слово из названия работы на том языке, на котором опубликована данная работа. Допускается сокращение, если слово очень длинное;
				\item Перечисляются Заглавные первые буквы всех остальных слов названия работы за исключением служебных слов, таких как предлоги, частицы, артикли и т.п.;
				\item Ставится дефис;
				\item Указывается год издания работы;
				\item Указывается 2-3 буквенный код, обозначающий тип работы, например:\\
				кн или bk - книга\\
				ст или art - статья\\
				\uline{Например:}\\
				Мартынов.В.В.СемиологОИ-1974кн\\
				Мартынов В. В., Семиологические основы информатики, Минск, 1974\\
				
				Golenkov.V.V..MethodsTECCS-2019art\\
				Methods and tools for ensuring compatibility of computer systems / V. Golenkov [et al.] // Открытые семантические технологии проектирования интеллектуальных систем = Open Semantic Technologies for Intelligent Systems (OSTIS-2019) : материалы международной научно-технической конференции, Минск, 21 - 23 февраля 2019 г. / Белорусский государственный университет информатики и радиоэлектроники; редкол.: В. В. Голенков (гл. ред.) [и др.]. - Минск, 2019. - С. 25 - 52.\\
				
				Идентификаторы электронных и прочих ресурсов формируются аналогичным образом, с учетом того, что опускается год издания и фамилии авторов, а также ставится буквенный код эл для обозначения электронного ресурса:\\
				
				МетасистIMS-эл \\
				Cypher-2017эл 
			\end{scnitemizeii}
		\end{scnitemizeiii}
	\end{scnitemizeii}
	\item \textbf{Основные принципы работы с транслятором в SCs}
	\item \textbf{Основные принципы навигации по БЗ через sc-web}
\end{scnitemize}
\end{SCn}