\scnsegmentheader{Что нужно знать для старта}
\scnstartsubstruct

\scnheader{начало работы}

\scnrelfromvector{этапы подготовки}{
	Изучить Youtube-канал с вводными видео о \textit{Технологии OSTIS}. Для начала рекомендуются к просмотру \textit{Введение в Технологию OSTIS} и \textit{Введение в SC-код}.;
	Изучить опубликованную версию \textit{Стандарта OSTIS-2021}.\\
	\scnrelfromvector{необходимые разделы}{
		Принципы построения и структура \textit{Стандарта OSTIS};
		Описание внутреннего языка \textit{ostis-систем};
		Предметная область и онтология внешних идентификаторов знаков, входящих в информационные конструкции внутреннего языка \textit{ostis-систем};
		Описание языка графического представления информационных конструкций в \textit{ostis-системах};
		Описание языка линейного представления информационных конструкций в \textit{ostis-системах};
		Описание языка структурированного представления информационных конструкций в \textit{ostis-системах};
		Описание общих принципов оформления внутреннего и внешнего представления информационных конструкций в \textit{ostis-системах};
		Предметная область и онтология смыслового представления информации};
	Целесообразно внимательно ознакомиться с актуальной версией \textit{Оглавления Стандарта OSTIS}, которая значительно больше, чем оглавление опубликованной версии \textit{Стандарта OSTIS-2021}. Наиболее актуальная и полная версия \textit{Оглавления} формируется при компиляции текста \textit{Стандарта из однодников} (см. \textit{Основные принципы работы с исходниками Стандарта OSTIS посредством Github}). В актуальную версию включены как разделы, уже имеющие некоторое наполнение, так и разделы, которые планируются к разработке в будущем (планы развития стандарта). Таким образом, любой желающий может выбрать наиболее интересное для себя направление для подключения к разработке \textit{Стандарта}.;
	Целесообразно внимательно ознакомиться с перечнем ключевых знаков, приведенный в \textit{Титульной спецификации Стандарта}, и синонимов к ним, что позволит лучше понять терминологию, используемую в рамках стандарта, а следовательно и сам стандарт.
}





